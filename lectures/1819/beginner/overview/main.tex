\documentclass{article}
\usepackage[utf8]{inputenc}

\title{Machine Learning Club}
\author{ML Club Officers}
\date{September 12, 2018}

\usepackage{natbib}
\usepackage{graphicx}
\usepackage{amsmath}

\begin{document}

\maketitle

\section{Introduction}
Machine Learning Club is an organization where members can learn the theory and applications of machine learning, test their skills in intra-club contests and online competitions, and conduct machine learning research. We meet every Wednesday A Block in Room 67/68 and Galileo Commons.

\section{Officers}
\begin{itemize}
    \item Captain: Alan Zheng
    \item Captain: Sylesh Suresh
    \item Teaching Coordinator: Vinay Bhaip
    \item Teaching Coordinator: Joseph Rejive
\end{itemize}

Contact the club officers through Facebook Messenger (preferred) or at \texttt{tjmachinelearning@gmail.com}.

\section{Lectures}

Lectures will begin with standard machine learning topics before delving into deep learning. We cover not only classical machine learning and deep learning algorithms, but also new and exciting advances. As research comes out throughout the year, we will periodically discuss any interesting developments. This year we will be splitting the club into advanced and beginner's groups, meaning there will be two lecture series.

The following lecture schedule is the current plan for the school year for the beginner's group. The lecture schedule is subject to change, and the most up to date version can always be found at \texttt{tjmachinelearning.com/schedule/1819}.

\begin{tabular}{|r|l| }
  \hline
\textbf{Date} & \textbf{Lecture} \\
  \hline
9/12/18 & Intro to Machine Learning \\ 
9/19/18 & Random Half-Day \\ 
9/26/18 & Decision Trees \\ 
10/3/18 & Random Forests \\ 
10/10/18 & PSAT Day \\ 
10/17/18 & SVMs \\ 
10/24/18 & Neural Networks Intro \\ 
10/31/18 & Neural Networks: Forward and Backpropagation \\ 
11/7/18 & Neural Networks: Optimizations, Introduction to Keras \\ 
11/14/18 & Intro to TF, PyTorch, Google Colab \\ 
11/21/18 & Thanksgiving Break \\ 
11/28/18 & CNNs: Intro \\ 
12/5/18 & CNNs: Continuation and Demos \\ 
12/12/18 & Chillin'/TBD \\ 
12/19/18 & Naive Bayes and K-Nearest Neighbors \\ 
12/26/18 & Winter Break \\ 
1/2/19 & Winter Break \\ 
1/9/19 & K-Means Clustering \\ 
1/16/19 & Intro to RNNs + LSTM/GRU \\ 
1/23/19 & RNNs: Continuation \\ 
1/30/19 & Intro to NLP \\ 
2/6/19 & Intro to GANs \\ 
2/13/19 & GANs: Part 2 \\ 
2/20/19 & Even more GANs \\ 
2/27/19 & Intro to Reinforcement Learning and Gym \\ 
3/6/19 & Chillin'/TBD \\ 
3/20/19 & CapsuleNet \\ 
3/27 - 4/24 & Guest Lectures, Competitions, Research, More \\
5/1/19  & Spring Break \\ 
5/8 - 5/29 & TBD \\
6/5/19 & Elections \\
6/12/19 & Election Results \\
  \hline
\end{tabular}
\newline
The advanced lecture series schedule is still in development and will be communicated to the upper group shortly. We plan on reviewing competition strategy and the basics of ML first, and then delve into complex architecture.

\section{Quizzes and Exams}
Quizzes and exams will be given occasionally to ensure students learn and retain material. These, along with the In-House contests (see Section 5.1) will be used to rank students.

\section{Competitions}
\subsection{In-House Contests}
Machine Learning Club will be holding weekly in-house contests through Kaggle Classroom. Details on the specifics of Kaggle Classroom will be given alongside the first contest. Unless otherwise specified, contests will begin on Wednesday evening after Machine Learning Club and run until 11:59:00 P.M. on the following Tuesday. Students will be ranked based on their achievement in these contests.

\subsection{Outside Competitions}
As the year progresses, Machine Learning Club members can participate in Kaggle competitions (\texttt{kaggle.com/competitions}). Substantial prize money is awarded to winners of contests, however, students will be competiting against anyone in the entire world, so the probability of winning is extremely low. Nevertheless, Kaggle competitions are a valuable learning experience.

\section{Research}
Machine Learning Club is formalizing its research initiative. This year, anyone who is having trouble with an ML research problem, wants input on the feasibility and/or scope of potential projects, or is looking for general ML help can ask the officers and receive help during club time. The current officers have experience at the highest level of high school research. After the lecture schedule ends in March, students will have more time to work on research projects, applying the knowledge from the lectures to real-world data.

\section{The Website}
Most information is conveyed through the official Machine Learning Club website, \texttt{https://tjmachinelearning.com/}. Here, you can find the lectures in pdf and web form, along with any presentations, notes, rankings, or additional resources.

\section{First Day Form}
Go to \texttt{https://tjmachinelearning.com/} and click "Join Us Today" on your phone or computer. Fill out the form to get on the email list if you haven't already.

\end{document}
